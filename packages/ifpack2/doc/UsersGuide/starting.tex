This section is meant to get you using \ifpacktwo{} as quickly as possible.
\S\ref{sec:overview} gives a brief overview of \ifpacktwo{}.
\S\ref{sec:configuration_and_build} lists \ifpacktwo{}'s dependencies on other
\trilinos{} libraries and provides a sample cmake configuration line. Finally,
some examples of code are given in~\S\ref{sec:examples in code}.

\section{Overview of \ifpacktwo{}}
\label{sec:overview}
\ifpacktwo{} is a C++ linear solver library in the \trilinos{} project~\cite{Heroux2012}.
It originally began as a migration of \ifpack{} package capabilities to a new linear
algebra stack. While it retains some commonalities with the original package, it
has since diverged significantly from it and should be treated as completely
independent package.

\ifpacktwo{} only works with \tpetra{}~\cite{TpetraURL} matrix,
vector, and map types. Like \tpetra{}, it allows for different ordinal
(index) and scalar types. \ifpacktwo{} was designed to be efficient on a wide
range of computer architectures, from workstations to supercomputers~\cite{Lin2014}.
It relies on the ``MPI+X" principle, where ``X'' can be threading or
CUDA\@. The ``X'' portion, node-level parallelism, is controlled by a node
template type. Users should refer to \tpetra{}'s documentation for information
about node and device types.

\ifpacktwo provides a number of different solvers, including
\begin{itemize}
  \item Jacobi, Gauss-Seidel, polynomial, distributed relaxation;
  \item domain decomposition solvers;
  \item incomplete factorizations.
\end{itemize}
This list of solvers is not exhaustive. Instead, references for further
information are provided throughout the text. There are many excellent
references for iterative methods, including~\cite{Saad2003}.

Complete information on available capabilities and options can be found
in~\S\ref{sec:options}.

\section{Configuration and Build}\label{sec:configuration_and_build}

\ifpacktwo{} requires a C++11 compatible compiler for compilation. The
minimum required version of compilers are GCC (4.7.2 and later),
Intel (13 and later), and clang (3.5 and later).

\subsection{Dependencies}

Table~\ref{tab:dependencies} enumerates the dependencies of \ifpacktwo. Certain
dependencies are optional, whereas others are required.  Furthermore,
\ifpacktwo's tests depend on certain libraries that are not required if you only
want to link against the \ifpacktwo library and do not want to compile its
tests. Additionally, some functionality in \ifpacktwo{} may depend on other
Trilinos packages (for instance, \amesostwo{}) that may require additional
dependencies. We refer to the documentation of those packages for a full list of
dependencies.

\begin{table}[ht]
  \centering
  \begin{tabular}{p{3.5cm} c c c c}
    \toprule
    \multirow{2}{*}{Dependency} & \multicolumn{2}{c}{Library} & \multicolumn{2}{c}{Testing} \\
    \cmidrule(r){2-3} \cmidrule(l){4-5} & Required & Optional & Required & Optional  \\
    \midrule
    % \belos                       & $\times$ &          & $\times$ & \\
    \teuchos                     & $\times$ &          & $\times$ & \\
    \tpetra                      & $\times$ &          & $\times$ & \\
    \tpetrakernels               & $\times$ &          &          & \\
    \amesostwo                   &          & $\times$ &          & $\times$  \\
    \galeri                      &          &          &          & $\times$  \\
    \xpetra                      &          & $\times$ &          & $\times$  \\
    \zoltantwo                   &          & $\times$ &          & $\times$  \\
    \textsc{ThyraTpetraAdapters} &          & $\times$ &          & \\
    \textsc{ShyLUBasker}         &          & $\times$ &          & $\times$ \\
    \textsc{ShyLUHTS}            &          & $\times$ &          & $\times$ \\
    \midrule
    % BLAS                         & $\times$ &          & $\times$ & \\
    % LAPACK                       & $\times$ &          & $\times$ & \\
    MPI                          &          & $\times$ &          & $\times$  \\
    % Cholmod                      &          & $\times$ &          & $\times$  \\
    % SuperLU 4.3                  &          & $\times$ &          & $\times$  \\
    % QD                           &          & $\times$ &          & $\times$  \\
    \bottomrule
  \end{tabular}
  \caption{\label{tab:dependencies}\ifpacktwo{}'s required and optional dependencies,
    subdivided by whether a dependency is that of the \ifpacktwo{}{} library itself
    (\textit{Library}), or of some \ifpacktwo{}{} test (\textit{Testing}). }
\end{table}

\amesostwo and \superlu are necessary if you want to use either a sparse direct
solve or ILUTP as a subdomain solve in processor-based domain decomposition.
\zoltantwo and \xpetra are necessary if you want to reorder a matrix (e.g.,
reverse Cuthill McKee).

\subsection{Configuration}
The preferred way to configure and build \ifpacktwo{} is to do that outside of the source directory.
Here we provide a sample configure script that will enable \ifpacktwo{} and all of its optional dependencies:
\begin{lstlisting}
  export TRILINOS_HOME=/path/to/your/Trilinos/source/directory
  cmake -D BUILD_SHARED_LIBS:BOOL=ON \
        -D CMAKE_BUILD_TYPE:STRING="RELEASE" \
        -D CMAKE_CXX_FLAGS:STRING="-g" \
        -D Trilinos_ENABLE_EXPLICIT_INSTANTIATION:BOOL=ON \
        -D Trilinos_ENABLE_TESTS:BOOL=OFF \
        -D Trilinos_ENABLE_EXAMPLES:BOOL=OFF \
        -D Trilinos_ENABLE_Ifpack2:BOOL=ON \
        -D Ifpack2_ENABLE_TESTS:STRING=ON \
        -D Ifpack2_ENABLE_EXAMPLES:STRING=ON \
        -D TPL_ENABLE_BLAS:BOOL=ON \
        -D TPL_ENABLE_MPI:BOOL=ON \
        ${TRILINOS_HOME}
\end{lstlisting}

\noindent
More configure examples can be found in \texttt{Trilinos/sampleScripts}.
For more information on configuring, see the \trilinos Cmake Quickstart guide \cite{TrilinosCmakeQuickStart}.

\section{Interface to \ifpacktwo{} methods}
All \ifpacktwo operators inherit from the base class
\texttt{Ifpack2::Preconditioner}. This in turn inherits from
\texttt{Tpetra::Operator}. Thus, you may use an \ifpacktwo operator anywhere
that a \texttt{Tpetra::Operator} works. For example, you may use \ifpacktwo operators
directly as preconditioners in \trilinos' \belos package of iterative solvers.

You may either create an \ifpacktwo operator directly, by using the class and
options that you want, or by using \texttt{Ifpack2::Factory}. Some of
\ifpacktwo preconditioners only accept a \texttt{Tpetra::\\CrsMatrix} instance as
input, while others also may accept a \texttt{Tpetra::RowMatrix} (the base class
of \texttt{Tpetra::CrsMatrix}). They will decide at run time whether the input
\texttt{Tpetra::RowMatrix} is an instance of the right subclass.

\texttt{Ifpack2::Preconditioner} includes the following methods:
\begin{itemize}
  \item \texttt{initialize()}

    Performs all operations based on the graph of the matrix (without
    considering the numerical values).

  \item \texttt{compute()}

    Computes everything required to apply the preconditioner, using the matrix's
    values.

  \item \texttt{apply()}

    Applies or ``solves with'' the preconditioner.
\end{itemize}
Every time that \texttt{initialize()} is called, the object destroys all the
previously allocated information, and reinitializes the preconditioner. Every
time \texttt{compute()} is called, the object recomputes the actual values of the
preconditioner.

An \ifpacktwo preconditioner may also inherit from
\texttt{Ifpack2::CanChangeMatrix} class in order to express that users can
change its matrix (the matrix that it preconditions) after construction using a
\texttt{setMatrix} method.  Changing the matrix puts the preconditioner back in
an ``pre-initialized'' state.  You must first call \texttt{initialize()}, then
\texttt{compute()}, before you may call \texttt{apply()} on this preconditioner.
Depending on the implementation, it may be legal to set the matrix to null. In
that case, you may not call \texttt{initialize()} or \texttt{compute()} until
you have subsequently set a nonnull matrix.

\textbf{Warning.} If you are familiar with the \ifpack package~\cite{ifpack}, please be aware
that the behaviour of the \ifpacktwo preconditioner is different from \ifpack.
In \ifpack, the \texttt{ApplyInverse()} method applies or ``solves with'' the
preconditioner $M^{-1}$, and the \texttt{Apply()} method ``applies'' the
preconditioner $M$. In \ifpacktwo, the \texttt{apply()} method applies or
``solves with'' the preconditioner $M^{-1}$. \ifpacktwo has no method comparable
to \ifpack's \texttt{Apply()}.

\section{Example: \ifpacktwo preconditioner within \belos}\label{sec:examples in code}

The most commonly used scenario involving \ifpacktwo{} is using one of its
preconditioners preconditioners inside an iterative linear solver. In
\trilinos{}, the \belos{} package provides important Krylov subspace methods (such
as preconditioned CG and GMRES).

At this point, we assume that the reader is comfortable with \teuchos{} referenced-counted
pointers (RCPs) for memory management (an introduction to RCPs can be found
in~\cite{RCP2010}) and the \parameterlist class~\cite{TeuchosURL}.

First, we create an \ifpacktwo{} preconditioner using a provided \parameterlist
\begin{lstlisting}[language=C++]
 typedef Tpetra::CrsMatrix<Scalar, LocalOrdinal, GlobalOrdinal, Node>
   Tpetra_Operator;

 Teuchos::RCP<Tpetra_Operator> A;
 // create A here ...
 Teuchos::ParameterList paramList;
 paramList.set( "chebyshev: degree", 1 );
 paramList.set( "chebyshev: min eigenvalue", 0.5 );
 paramList.set( "chebyshev: max eigenvalue", 2.0 );
 // ...
 Ifpack2::Factory factory;
 RCP<Ifpack2::Ifpack2Preconditioner<> > ifpack2Preconditioner;
 ifpack2Preconditioner = factory.create( "CHEBYSHEV", A )
 ifpack2Preconditioner->setParameters( paramList );
 ifpack2Preconditioner->initialize();
 ifpack2Preconditioner->compute();
\end{lstlisting}

Besides the linear operator $A$, we also need an initial guess vector for the
solution $X$ and a right hand side vector $B$ for solving a linear system.
\begin{lstlisting}[language=C++]
 typedef Tpetra::Map<LocalOrdinal, GlobalOrdinal, Node> Tpetra_Map;
 typedef Tpetra::MultiVector<Scalar, LocalOrdinal, GlobalOrdinal, Node>
   Tpetra_MultiVector;

 Teuchos::RCP<const Tpetra_Map> map = A->getDomainMap();

 // create initial vector
 Teuchos::RCP<Tpetra_MultiVector> X =
   Teuchos::rcp( new Tpetra_MultiVector(map, numrhs) );

 // create right-hand side
 X->randomize();
 Teuchos::RCP<Tpetra_MultiVector> B =
   Teuchos::rcp( new Tpetra_MultiVector(map, numrhs) );
 A->apply( *X, *B );
 X->putScalar( 0.0 );
\end{lstlisting}
To generate a dummy example, the above code first declares two vectors. Then, a
right hand side vector is calculated as the matrix-vector product of a random vector
with the operator $A$. Finally, an initial guess is initialized with zeros.

Then, one can define a \texttt{Belos::LinearProblem} object where the
\texttt{ifpack2Preconditioner} is used for left preconditioning.
\begin{lstlisting}[language=C++]
 typedef Belos::LinearProblem<Scalar, Tpetra_MultiVector, Tpetra_Operator>
   Belos_LinearProblem;

 Teuchos::RCP<Belos_LinearProblem> problem =
   Teuchos::rcp( new Belos_LinearProblem( A, X, B ) );
 problem->setLeftPrec( ifpack2Preconditioner );
 bool set = problem.setProblem();
\end{lstlisting}

Next, we set up a \belos{} solver using some basic parameters.
\begin{lstlisting}[language=C++]
 Teuchos::RCP<Teuchos::ParameterList> belosList =
   Teuchos::rcp(new Teuchos::ParameterList);
 belosList->set( "Block Size", 1 );
 belosList->set( "Maximum Iterations", 100 );
 belosList->set( "Convergence Tolerance", 1e-10 );
 belosList->set( "Output Frequency", 1 );
 belosList->set( "Verbosity", Belos::TimingDetails + Belos::FinalSummary );

 Belos::SolverFactory<Scalar, Tpetra_MultiVector, Tpetra_Operator> solverFactory;
 Teuchos::RCP<Belos::SolverManager<Scalar, Tpetra_MultiVector, Tpetra_Operator> >
   solver = solverFactory.create( "Block CG", belosList );
 solver->setProblem( problem );
\end{lstlisting}

Finally, we solve the system.
\begin{lstlisting}[language=C++]
 Belos::ReturnType ret = solver.solve();
\end{lstlisting}

It is often more convenient to specify the parameters as part of an XML-formatted options file.
Look in the subdirectory {\tt Trilinos/packages/ifpack2/test/belos} for examples of this.

This section is only meant to give a brief introduction on how to use
\ifpacktwo{} as a preconditioner within the \trilinos{} packages for iterative
solvers. There are other, more complicated, ways to use to work with
\ifpacktwo{}. For more information on these topics, the reader may refer to the
examples and tests in the \ifpacktwo{} source directory
(\texttt{Trilinos/packages/ifpack2}).
